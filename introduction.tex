\chapter*{Вступ}

Доступність та відкритість даних, що мають суттєве суспільне значення, стали світовим трендом.
За цих умов питання забезпечення доступу до публічної інформації набувають особливої ваги для регіональних та місцевих рівнів публічного управління.
Адже, незважаючи на напрацьований протягом останніх років досвід,
публічні службовці нині мають розв’язувати нові комплексні та складні проблеми за цим напрямом діяльності, у тому числі з
використанням сучасних інформаційно-комунікаційних технологій.
Постійне удосконалення чинної нормативно-правової бази та
організаційної структури органів влади спонукає запроваджувати та
використовувати у практиці публічного управління та адміністрування нові технологічні інструменти, спрямовані на поліпшення
інформаційних обмінів між органами влади та суспільством.

\textbf{Актуальнiсть} теми дипломної роботи тiсно пов’язана iз зростаючою
тенденцiєю вiдкриття даних у світі та Україні.
У країнах ЄС вiдкриття державних даних вже понад десятиріччя набуває широкого
поширення задля боротьби з бюрократiєю та корупцією.

За результатами дослiдження Open Data Barometer Україна зайняла 2 мiсце у світі     за темпами
вiдкриття держваних даних.
Наша країна розвивається у цьому напрямку тiльки останнi декiлька рокiв,
тому питання обробки цих даних є досить нагальним.

Портал вiдкритих даних — єдиний державний веб-портал вiдкритих даних,
створений з метою зберiгання публiчної iнформацiї у формi вiдкритих даних
та забезпечення надання доступу до неї широкому колу осiб за принципами,
визначеними у Мiжнароднiй хартiї вiдкритих даних, до якої Україна приєдналася
у жовтнi 2016 року. Портал створено на вимогу Закону України «Про доступ до
публiчної iнформацiї» та постанови Кабiнету Мiнiстрiв України вiд 21 жовтня
2015 року No 835 «Про затвердження Положення про набори даних,
якi пiдлягають оприлюдненню у формi вiдкритих даних».

Водночас бiльшiсть громадян України зустрiчаються з проблемою неможливостi
знаходження iнформацiї в обробленому для користувача виглядi, саме тому
\textbf{об'єктом дослідження} цієї роботи було обрано процес ефективної обробки
великих масивiв даних, що зберiгаються на єдиному порталi даних України.
А оскільки, не існує єдиного способу обробки різних типів даних, за
\textbf{предмет дослідження} було обрано  вивчення методiв ефективної обробки
великих масивiв даних на прикладi даних про український транспорт.

Очiкуваним результатом i \textbf{метою} дипломної роботи є розробка прототипу
швидкісного веб-сервiса, що надасть чiтку iнформацiю про український транспорт
за державним номерним знаком.

\textbf{Практична значимість} дипломної роботи полягає в можливостi
застосування її результату на практицi з метою надання користувачам рiзних
iнструментiв для пошуку транспортних засобiв.

У ходi виконання дипломної роботи на першому етапi буде розроблена
архiтектура веб-сервiсу, наступним кроком буде реалiзацiя та тестування
програмного забезпечення.
Кінцевим результатом роботи буде чат-бот з використанням розробленого
програмного продукту, аби продемонструвати використання сервіса на прикладі.

Серед \textbf{методів розроблення} будуть задіяні комп’ютерне моделювання та розробка програмного
продукту на основі ітеративної моделі. У якості \textbf{інструментів розроблення} будуть використовуватися
мова програмування Go, реляціяні бази даних та
єдиний державний портал відкритих даних.
