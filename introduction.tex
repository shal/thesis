\chapter*{Вступ}

Доступність та відкритість даних, що мають суттєве суспільне значення, стали світовим трендом.
За цих умов питання забезпечення доступу до публічної інформації набувають особливої ваги для регіональних та місцевих рівнів публічного управління.
Адже, незважаючи на напрацьований протягом останніх років досвід,
публічні службовці нині мають розв'язувати нові комплексні та складні проблеми за цим напрямом діяльності, у тому числі з
використанням сучасних інформаційно-комунікаційних технологій.
Постійне удосконалення чинної нормативно-правової бази та
організаційної структури органів влади спонукає запроваджувати та
використовувати у практиці публічного управління та адміністрування нові технологічні інструменти, спрямовані на поліпшення
інформаційних обмінів між органами влади та суспільством.

\textbf{Актуальність} теми дипломної роботи тісно пов'язана із висхідною
тенденцією відкриття даних у світі та Україні.
У країнах ЄС тема відкритості державних даних вже понад десятиріччя набуває широкого
поширення задля боротьби з бюрократією та корупцією.

За результатами дослідження Open Data Barometer Україна зайняла 2 місце у світі за темпами
відкриття державних даних.
Наша країна розвивається у цьому напрямку тільки останні декілька років,
тому питання обробки цих даних є досить нагальним.

Портал відкритих даних — єдиний державний веб-портал відкритих даних,
створений з метою зберігання публічної інформації у формі відкритих даних
та забезпечення надання доступу до неї широкому колу осіб за принципами,
визначеними у Міжнародній хартії відкритих даних, до якої Україна приєдналася
у жовтні 2016 року. Портал створено на вимогу Закону України «Про доступ до
публічної інформації» та постанови Кабінету Міністрів від 21 жовтня
2015 року No 835 «Про затвердження Положення про набори даних,
які підлягають оприлюдненню у формі відкритих даних».

Водночас більшість громадян України зустрічаються з проблемою неможливості
знаходження інформації в обробленому для користувача вигляді, саме тому
\textbf{об'єктом дослідження} цієї роботи було обрано процес ефективної обробки
великих масивів даних, що зберігаються на єдиному порталі даних України.
А оскільки, не існує єдиного способу обробки різних типів даних, за
\textbf{предмет дослідження} було обрано  вивчення методів ефективної обробки
великих масивів даних на прикладі даних про український транспорт.

Очікуваним результатом i \textbf{метою} дипломної роботи є розробка прототипу
швидкісного веб-сервіса, що надасть чітку інформацію про український транспорт
за державним номерним знаком.

\textbf{Практична значущість} дипломної роботи полягає в можливості
застосування її результату на практиці з метою надання користувачам різних
інструментів для пошуку транспортних засобів.

У ході виконання дипломної роботи на першому етапі буде розроблена
архітектура веб-сервісу, наступним кроком буде реалізація та тестування
програмного забезпечення.
Кінцевим результатом роботи буде чат-бот з використанням розробленого
продукту, аби продемонструвати використання сервісу на прикладі.

Серед \textbf{методів розроблення} будуть долучені комп’ютерне моделювання та розробка програмного
продукту на основі ітеративної моделі. У якості \textbf{інструментів розроблення} будуть використовуватися
мова програмування Go, реляційні бази даних та
єдиний державний портал відкритих даних.
