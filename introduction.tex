\chapter*{Вступ}

Актуальність теми дипломної роботи тісно пов'язана із всесвітньою тенденцією відкритих даних. У країнах Європейського Союзу відкриття державних даних вже понад десяти років набуває широкого поширення задля боротьби з бюрократією.

За результатами дослідження Open Data Barometer за 2017 рік Україна зайняла 47 місце в рейтингу розвитку відкритих даних. Наша країна розвивається у цьому напрямку тільки останні декілька років, тому питання обробки цих даних є досить нагальним.

«Портал відкритих даних» — єдиний державний веб-портал відкритих даних, створений з метою зберігання публічної інформації у формі відкритих даних та забезпечення надання доступу до неї широкому колу осіб за принципами, визначеними у Міжнародній хартії відкритих даних, до якої Україна приєдналася у жовтні 2016 року. Портал створено на вимогу Закону України «Про доступ до публічної інформації» та постанови Кабінету Міністрів України від 21 жовтня 2015 року № 835 «Про затвердження Положення про набори даних, які підлягають оприлюдненню у формі відкритих даних».

В той самий час більшість громадян України зустрічаються з проблемою неможливості знаходження інформації в обробленому для користувача вигляді. На нашу думку, так відбувається через погану якість та різні форми представлення цих даних. Оскільки, не існує одного способу обробки різних даних, нами було обрано одну з найцікавіших для користувачів сферу — Транспорт.

Очікуваним результатом і метою дипломної роботи є розробка прототипу веб-сервісу і веб-інтерфейсу, що надають просту та чітку інформацію про транспорт за державним номерним знаком.

Практична значущість дипломної роботи полягає в можливості застосування її результату на практиці з метою надання користувачам різних інструментів для пошуку транспортних засобів.

У ході виконання дипломної роботи на першому етапі буде розроблена архітектура веб-сервісу, наступним етапом буде реалізація та тестування програмного забезпечення, кінцевим результатом буде Telegram чат-бот з використанням розробленого веб-сервісу як зовнішній сервіс, для того щоб продемонструвати як його використовувати на прикладі.
