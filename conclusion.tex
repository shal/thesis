\chapter*{Висновки}

На прикладі відкритих даних про українські транспортні засоби,
що публікуються на єдиному державному порталі відкритих даних, були
вивчені ефективні методи обробки великих масивів даних та
детально розібрано процес обробки відкритих даних з цього порталу.
Запропонований алгоритм роботи з даними на єдиному державному порталi вiдкритих даних.

Отже, можна ствержувати, що сфера даних має великий потенціал у нащій країні,
проте якість даних, що викладаються на порталі, нажаль, є досить низькою.
Стало зрозуміло, що формат документів повинен бути змінений якнайшвидше,
оскільки із зростанням кількості даних - зростає складність та час їх обробки.

Таким чином, вища якість даних повинна призвести до більших часових та грошових
інвестицій у сфері оброки даних в Україні,
а отже можна отримати додаткові 1.5 млдр. доларів у бюджет країни.

Результатом проведеної дипломної роботи є спроектований та створений програмний продукт.
Для усіх мешканців України сервіс є простим і зручним інструментом пошуку
інформації про автотранспорт.

Під час дослідження ми дізналися, що Україна стає частиною єдиного Європейського інформаційного простору
і ще більше наближається до інтеграції з Євросоюзом.
Завдяки відкритим даним Україна стає більш привабливою для західних інвестицій.
Український уряд намагається розвивати сферу та заохочує нових спеціалістів за допомогою створення відповідних конкурсів та
тендерів на розробку додатків з використанням відкритих даних.

В ході написання роботи на першому етапі була розроблена архітектура
майбутнього сервісу з урахуванням всіх вимог і побажань.
На другому етапі програмний продукт був реалізований в суворій відповідності з
встановленими вимогами та створеної архітектурою.
Під час реалізації продукту були використані сучасні технології.
Так само було приділено увагу питанню подальшого
можливого коригування вихідного продукту. Продукт вийшов
досить гнучким на випадок бажання замовника змінити середу або
тип бази даних. У ході реалізації було приділено увагу
використанню надійних технологій збереження даних.
На останньому етапі увесь функціонал був перевірений в ручному режимі.

Розроблений програмний продукт має вiдкритий код, та може використовуватися у комерцiйних цiлях будь-якого пiдприємця чи компанiї.
Система була розгорнута, тож будь-який користувач системи Telegram може її використовувати.

Було вдало оброблено понад 9 млн записів з документів формату CSV,
які були викладені Міністерством Внутрішніх Справ України у якості відкритих даних.
Загальний час оброки усіх файлів займає менш як 10 хвилин.
На жаль, у цих даних відсутня інформація щодо ідентифікаційних
номерів транспортних засобів, тож не можливо
прослідкувати історію транспорту, а лише побачити поточні дані.

Підбиваючи підсумки виконаної роботи, можна стверджувати, що мета
дипломної роботи повністю виконана - прототип швидкiсного веб-сервiса,
що надає iнформацiю про український транспорт за державним номерним знаком створений,
вимоги враховані і реалізовані.