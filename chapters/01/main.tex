\chapter{Відкриті дані}

\section{Поняття відкритих даних}

Переважна більшість організацій користується визначенням, що було дано всесвітньою організацією «Open Knowledge Foundation» \cite{OpenDefinition}.
Це визначення відкритості подає точне значення терміну «відкрите» стосовно знань,
сприяючи стійким громадам, в яких кожен може взяти участь
і де можливість взаємодії досягає максимуму.

Повне значення відкритості є надто чітким для цієї роботи \cite{OpendataHandBook},
виділимо з визначення основні риси відкритості:

\textbf{Доступність.}
Дані повинні бути доступні у повній мірі. Ресурси на їх отримання повинні бути
у розумних межах, переважно шляхом завантаження через Інтернет.
Дані також повинні бути доступні у зручній і змінюваній формі.

\textbf{Повторне використання та розповсюдження.}
Дані повинні надаватися на умовах, що дозволяють повторне
використання та розповсюдження, включаючи
використання з іншими наборами даних.

\textbf{Загальна участь.}
Кожен повинен мати змогу використовувати та розповсюджувати дані.
Будь-які обмеження щодо сфер діяльності, осіб чи груп повинні бути відсутніми.
Наприклад, не комерційні обмеження, які запобігають комерційному використанню,
або обмежується використання для певних цілей (наприклад, тільки в освіті), не допускаються.

Якщо вам цікаво, чому важливо дати чітке визначення відкритості, є проста відповідь: сумісність.
Сумісність означає здатність різних систем та організацій взаємодіяти разом \cite{BigDataOpenData}.
У даному випадку це можливість взаємодіяти або змішувати різні набори даних.
Сумісність є дуже важливою, оскільки дозволяє різним компонентам працювати разом.

Здатність до складання й підключення компонентів має важливе
значення для побудови великих, складних систем.
Без сумісності це стає неможливим, про що свідчить найвідоміший міф про
Вавилонську вежу, де проблема взаємодії призвела до повного краху.

Ми бачимо схожий випадок по відношенню до даних.
Сутність сумісності даних полягає у тому, що один фрагмент відкритих
даних може бути використаний разом з будь-яким іншим фрагментом відкритих даних.

Сумісність — це ключ до реалізації основних переваг
відкритості. Різко зростає здатність об'єднувати набори
різноманітних даних і тим самим розробляти більші й кращі
продукти й послуги.

Надання чіткого визначення відкритості гарантує, що при отриманні
двох відкритих наборів даних з двох різних джерел, ми будемо мати
змогу об'єднати їх разом, іншими словами ми не потрапимо у ситуацію
де багато наборів даних, проте взагалі неможливо об'єднати їх у великі
системи, які у свою чергу мають найбільшу цінність для бізнесу та подальшої обробки.

\section{Доступ до відкритих даних}

Сучасний етап суспільного розвитку позначений динамічним
проникненням інформаційних технологій в усі сфери людської
діяльності. За таких умов недостатня увага органів публічної влади
до характеру, змісту інформаційних обмінів з громадськістю, більше
того – їх недооцінка, можуть зумовити дисбаланс у відносинах влади
й громадськості \cite{TheGlobalImpact}.
З одного боку, це може загрожувати збільшенням
кількості нерезультативних і неефективних рішень, з іншого –
зниженням ступеня підтримки органів публічної влади суспільством, а
то й відкритим протистоянням. Унаслідок такого стану справ значна
кількість суспільно важливих проблем залишаться невирішеними,
належним чином не будуть прогнозовані та попереджені нові проблемні ситуації.
Для нормального функціонування описаної системи взаємовідносин важливо
досягти такого стану речей, коли інформаційні
потоки між органами публічної влади та громадськістю циркулюють
максимально безперешкодно. Така модель може бути реалізована,
коли кожен громадянин матиме реальну можливість на отримання
гарантованої законом повної, правдивої та всебічної інформації про
функціонування органів влади, про їх плани, можливі напрями дій,
стан окремих сфер суспільного життя, використання державних та
комунальних ресурсів \cite{FakeOpenDataEcosystem}.

За останні роки було ухвалено міжнародно-правові документи,
які визнають право на доступ до інформації основоположним правом
людини, а представники Організації Об’єднаних Націй, Ради Європи,
ОБСЄ у своїх доповідях звертають увагу на те, що право на доступ
до інформації є необхідною умовою для участі громадян в ухваленні
державних рішень, запорукою розвитку демократії та фундаментом у
боротьбі з корупцією \cite{EGAPOpenDataBook}.

У європейських державах діяльність, що пов'язана з роботою
з публічною інформацією, в цілому ґрунтується на принципі
максимальної відкритості, який полягає у тому, що державні органи
зобов'язані розкривати інформацію, а кожний член суспільства має
відповідне право отримувати її. При цьому обов'язок державних органів
щодо оприлюднення інформації, що має особливе суспільне значення,
здебільшого реалізується не лише як реакція на вимогу надати
інформацію у відповідь на запити, але й шляхом оприлюднення та
широкого розповсюдження документів, які становлять особливий
суспільний інтерес, залежно від наявних ресурсів і можливостей.
Це, зокрема, випливає з положень статті 10 Конвенції Ради Європи
про доступ до офіційних документів, якою встановлено, що органам
державної влади належить із власної ініціативи і в тих випадках, коли
це виправдано, вживати всіх необхідних заходів щодо опублікування
офіційних документів, які перебувають в їхньому розпорядженні, в
інтересах підвищення прозорості та ефективності органів державного
управління та сприяння інформованому залученню громадськості у
справи, які становлять суспільний інтерес

Аналогічні положення відображені й у національному законодавстві.
Так, у Республіці Болгарія законом про доступ до публічної інформації
саме з метою максимального полегшення доступу
до публічної інформації, передбачено обов'язок щодо
розкриття публічної інформації державними органами про свою
діяльність шляхом її регулярної публікації або з використанням
інших форм оприлюднення.
Також принцип сприяння доступу до
інформації реалізується і в такий спосіб, що усі державні органи
зобов'язані створити відкриті й доступні внутрішні
системи забезпечення права громадян на отримання інформації.
Для прикладу, Уряд Словенії зобов'язав усі державні установи створити на своїх
веб-сайтах та регулярно оновлювати каталоги публічної інформації.
Тобто, в питаннях забезпечення доступу до інформації за кордоном,
зокрема в Європі, пріоритетом є проактивна, а не реактивна позиція
органів влади.

Одним з яскравих прикладів застосування інноваційних механізмів
забезпечення доступу до публічної інформації в умовах розвитку
інформаційного суспільства є Фінляндія, де давня традиція
відкритості влади, поєдналася з інтенсивним розвитком ІКТ. Як
наслідок, політика відкритості та можливість електронного доступу
до інформації є основною причиною низької корупції у Фінляндії,
факти корупційних діянь у країні трапляються скоріше як винятки,
ніж як закономірність. Отже, Фінляндія є прикладом превентивного
забезпечення громадськості публічною інформацією.

Як свідчить закордонна практика, з налагодженням діяльності із
забезпечення доступу до публічної інформації в органах публічної
влади з часом зменшується кількість звернень про доступ до такої
інформації. При цьому зацікавленість нею не знижується – просто
громадяни, журналісти, окремі організації починають отримувати
необхідну інформацію з джерел, що стають дедалі доступнішими.
Зокрема, про це свідчить досвід Болгарії, де за останні п’ять років
кількість запитів на інформацію до органів влади постійно зменшувалася.

Таким чином, європейський досвід забезпечення доступу до
публічної інформації свідчить про перспективну зміну тенденцій і в
українській практиці – з розширенням використання інформаційно-комунікаційних
технологій споживачі такої інформації зможуть самостійно отримувати доступ
до необхідних документів, що зменшить необхідність безпосередніх контактів
запитувача і службовця, підвищить ефективність управлінських процедур.

\section{Класифікація відкритих даних}

Для того, щоб зрозуміти, які можуть бути форми відкритих даних, ми також звернемося до
відомої класифікації «Five Star Open Data» \cite{5StarData}, де якість даних та рівень
відкритості визначається кількістю зірок від 1 до 5, чим більше – тим краще. Відкритість
даних залежить від способів доступу, форматів та кількості додаткових дій, які потрібні для
отримання кінцевої інформації, її обробки та збереження у власному сховищі або базі даних.

\textbf{Одну зірку} отримує будь-яка інформація вільно доступна через Інтернет в будь-якому
форматі. Під цю класифікацію підпадає файл в форматі PDF або інша копія
документу, на який веде пряме посилання на офіційному сайті державного органу. Якщо
цей файл можна відкрити на власному екрані, прочитати, роздрукувати та отримати звідти
потрібну інформацію, то це відкриті дані з однією зіркою.

\textbf{Дві зірки} отримує структурована інформація, яку можна обробляти автоматично, наприклад,
в форматах для веб-браузерів чи офісних програм (відкриті формати – TXT, HTML,
RSS; пропрієтарні формати, Excel – XLS, Word – DOC, RTF). Якщо дані знаходяться в тілі
вихідної веб-сторінки, але не мають чіткої структури, містять зайві елементи оформлення,
навігації, якщо дані потребують додаткових дій – спеціального розбору, то вони
вважаються «з двома зірками».

\textbf{Три зірки} може отримати інформація, представлена у відомих, добре описаних відкритих структурованих форматах
(наприклад, CSV, JSON, XML, YAML) і якщо автоматизована її
обробка не потребує від користувача особливих ліцензій та додаткових плат. До відкритих
форматів також відносяться пов'язані дані (HTML+RDFa) з узгодженою розміткою елементів
в атрибутах або текстові файли таблиць, поля яких
розділені табуляцією, комами, крапками з комою або іншими символами.

\textbf{Чотири зірки} надаються у випадку, якщо можна отримати первинні необроблені набори
відкритих даних у вигляді файлів (довідники, списки, таблиці у відкритому форматі, зліпок
бази даних, архів документів тощо) або фільтровані дані у запиті до API за вказаними
параметрами. Це дає змогу отримувати тільки потрібну інформацію, актуальну на момент
запиту, заощаджує ресурси та час користувача. Безумовно, API має бути описаний так само,
як і формати даних, а доступ до нього може бути анонімний без обмежень або з реєстрацією,
за вказаним ідентифікатором, лімітами на кількість одночасних запитів.

Останній рівень – \textbf{п’ять зірок} – надається інформації, якщо набори відкритих даних
пов'язані між собою (мають спільні довідники, класифікатори, ідентифікатори, посилання
між документами та іншими елементами) і представляють собою семантичну мережу,
що постійно оновлюється й змінюється відповідно до сучасних запитів.

Слід зазначити, що більшість даних представлених на єдиному державному порталі відкритих даних
має формат CSV та JSON, тобто оцінка якості даних за цією технологією поки не дуже висока.

\section{Формати даних}

В залежності від специфіки даних, їх розміру та тематики,
одні проекти відкритих даних створювались на базі наборів PDF чи DOC файлів,
таблиць XLS, що перетворювались на прості текстові таблиці CSV, а інші брали за
основу формат розмітки XML, проектували власні схеми XSD і використовували складні
структури.

Як свідчить остання статистика використання форматів відкритих даних,
найбільш поширений в світі формат (як по кількості, так і по об'єму даних) – PDF.
Для українських органів влади, де найбільш розповсюджені операційні системи Microsoft Windows,
переважають формати DOC та XLS. Разом з новими версіями офісних програм в Інтернет почали
з’являтися документи DOCX та XLSX, рідко ODF (Open Document Format).
Після поширення ініціативи відкриття державних даних та
створення порталів, кількість наборів в форматі XML та інших відкритих форматах почала
суттєво збільшуватись.

Необроблені дані, сформовані державними структурами за багато років, можуть бути досить
неоднорідними, а деякі набори навіть дублюються в різних форматах для зручності користування.

Серед доступних форматів відкритих даних, які можна автоматично обробляти
електронними засобами, є:

\begin{itemize}
  \item CSV – дані, розділені комами або іншими розділовими символами.
  \item JSON – формат, орієнтований на обробку складних структур.
  \item XML – універсальний текстовий формат розмітки.
\end{itemize}

\textbf{CSV} – текстовий відкритий формат, призначений для представлення масивів даних, де
кожний рядок – це запис таблиці, а значення окремих полів у рядку розділені спеціальними
символами, зазвичай комами. Щоб завантажити записи таблиці за найменуванням полів, додатково потрібно мати опис її структури –
назви та формат полів.
Більшість програм широко трактують цей формат і допускають використання інших розділових символів,
наприклад, табуляції (TSV) чи коми з крапкою.

\textbf{JSON} – текстовий відкритий формат, оснований на Javascript представлені
та призначений для обміну даними в мережі Інтернет між сервером та
клієнтом або сервером і сервером. Хоча він позиціюється, як незалежний від системи і
мови програмування, частіше за все використовується за допомогою програм на Javascript,
але як і інші текстові формати, легко читається людиною.

Найбільшу популярність JSON набув після створення інтерактивних веб-сторінок, дані до
яких через API передавались під час взаємодії користувача з елементами інтерфейсу.
За рахунок своєї лаконічності, на відміну від XML, простоті й швидкості
використання саме в програмах на Javascript, широкими можливостями в обробці даних
– рекурсивного перетворення в текстовий вигляд складних об'єктів,
формат активно використовується для формування «на льоту»
та передачі структур даних в Інтернет в різних
інформаційних системах і сервісах.

Проте існує велика кількість інших форматів даних, які є менш розповсюдженими чи орієнтованими на певну сферу даних,
наприклад відеодані. Дані такого незвичайного виду зазвичай зберігають в спеціалізованих форматах.

\textbf{XML} – найстаріший текстовий відкритий формат, створений в 1994 році та рекомендований
Консорціумом Всесвітньої павутини, як основний для обміну інформацією в Інтернет.
Гіпертекстова розмітка (HTML) – це один з різновидів XML. Разом з таблицями каскадних
стилів CSS, які формують зовнішній вигляд документів, вони є тими основними форматами,
що обумовлюють розвиток технологій.
Перевагами XML є простота та гнучкість розмітки, яка не вимагає формальних, фіксованих
назв тегів чи параметрів, і будь-який розробник може доповнювати та змінювати формат,
створювати власну схему XSD.

За довгий час існування XML на його базі було розроблено багато форматів і стандартів зі
схожим синтаксисом. Зазвичай цю групу форматів називають загальною назвою – XML,
тому що вони мають єдині механізми опису схем XSD,
перевірки правильності даних, доступ до елементів XPath та трансформації для автоматичного конвертування
у інші схеми чи формати (наприклад, альтернативні JSON та YAML) за допомогою мови
перетворення XSLT (eXtensible Stylesheet Language Transformations).

Використання формату XML (а саме LegalXML) у якості відкритого стандарту
нормативноправового документа – це сучасний спосіб забезпечити обмін інформацією (документами,
картками, довідниками тощо) між інформаційними системами або в межах однієї системи
при опрацюванні документів.

В одному файлі XML в текстовому вигляді, крім основних даних та тексту електронного
нормативного документа, можна розміщати метадані, вкладені файли,
необхідні структури чи довідники. Це дозволяє зручно не тільки зберігати,
передавати, обробляти документ, отримувати PDF версію для друку, формувати зміст чи
робити посилання на конкретну главу, статтю, пункт, підпункт тощо, а й автоматизовано
вносити зміни, підготовлені у вигляді, що дозволяє їх програмну обробку.

\section{Відкриті дані в Україні}

В багатьох країнах світу розвиток відкритих даних підтримується на державному рівні вже досить давно.
І цей процес включає створення відповідної законодавчої бази, виконавчих органів та інформаційних ресурсів \cite{BeyondTransparency}.

У цілому українське законодавство про доступ до публічної
інформації відповідає вимогам міжнародних стандартів про свободу доступу до інформації. 
В їх основу покладені принципи максимальної відкритості та доступності громадськості на отримання інформації, 
обов'язку оприлюднення інформації, яка має особливе значення,
надання відповіді у стислі терміни, забезпечення безкоштовної допомоги при оформленні запиту. Зокрема, Закон України «Про
доступ до публічної інформації» за міжнародним рейтингом забезпечення права на інформацію,
розроблений міжнародними організаціями «Access Info Europe» та «Centre for Law and Democracy»,
посів 8 місце серед 89 країн світу.

\textbf{Міжнародне визнання.}
Україна посіла друге місце серед країн, що досягнули найбільшого прогресу за чотири роки за
рівнем публікації та використання відкритих даних для підзвітності влади, розвитку
інновацій і соціального впливу. Про це свідчить звіт найбільш впливового світового рейтингу
Open Data Barometer за вересень 2018 року. Крім того, Україна посіла 17 сходинку серед 30
країн-лідерів, що взяли на себе конкретні зобов'язання щодо розвитку відкритих даних.
Це результат нашого уряду щодо забезпечення принципів прозорості та відкритості у державному
секторі, які стали першочерговими для України після приєднання у 2016 році до Міжнародної
хартії відкритих даних. У листопаді 2018 року Кабмін схвалив план дій з реалізації принципів
хартії, тому очікуємо подальшого динамічного розвитку сфери відкритих даних у 2019 році.

\textbf{Вплив на Економіку.}
Відкриті дані — це цінний ресурс, який допоможе посилити «цифрову» та «реальну» економіки країни, адже про це свідчать результати глобальних досліджень.
Очікується, що економіки країн Великої двадцятки зростуть на 1,1\% ВВП завдяки відкриттю пріоритетних наборів даних.
«Київська Школа Економіки» та «Open Data Institute» використовуючи авторитетну методологію дослідження Європейської комісії,
проаналізували економічний потенціал відкритих даних для України.
Ми стали першою країною за межами країн Великої двадцятки, де був проведений такий аналіз.
Дослідники виявили, що вже у 2017 році відкриті дані принесли нашій державі понад 700 млн. доларів
Якщо рух за відкриті дані в Україні й надалі набиратиме обертів,
ця цифра може зрости до 1,14 млрд доларів або 0,92\% ВВП до 2025 року.

\textbf{Українські дані на Європейському порталі.}
Україна стає частиною єдиного Європейського інформаційного простору і ще більше наближається до інтеграції з Євросоюзом.
З вересня 2018 року українські державні дані, як і дані країн ЄС, публікуються на «European Data Portal».
Портал містить понад 860 тис. наборів даних, що охоплюють 35 країн і 78 місцевих та національних порталів.
Що це дає Україні? По-перше — імідж.
Публікація наборів даних на «European Data Portal» — це потужний сигнал про відкритість
діяльності держави на рівні європейських країн.
По-друге — підвищення інвестиційної привабливості.
Європейські компанії можуть використовувати відкриті дані для отримання більш
детальної інформації для аналізу бізнес-потенціалу України.
Наприклад, «Open Corporates», найбільша відкрита база даних компаній у світі,
що містить інформацію про майже 140 млн компаній, була інтегрована з Єдиним державним реєстром юридичних осіб.
Це робить Український ринок більш привабливим для іноземних інвесторів, оскільки можна
швидко та зручно знаходити необхідні дані.

\textbf{Портал відкритих даних.}
У 2018 році Державне агентство з питань електронного урядування за підтримки проекту «Прозорість та підзвітність у державному управлінні та послугах» запустили модернізований державний портал відкритих даних.
Розробила проект команда «Opendatabot». Це найвідоміший стартап у сфері відкритих даних та один з найбільших користувачів порталу. Це унікальна світова практика, коли портал модернізують користувачі відкритих даних.
На оновленому порталі публікуються нові дані, які одразу можуть використовуватися бізнесом, громадськими організаціями та журналістами без додаткової обробки.
До того ж, тепер розпорядники можуть оновлювати дані автоматично, що дозволяє отримувати інформацію в реальному часі.

\textbf{Пріоритетні набори даних.}
Одне з найбільш вагомих досягнень 2018 року — відкриття пріоритетних наборів,
починаючи від транспортної сфери й закінчуючи даними місцевих бюджетів.

Найбільш популярним набором даних стали відомості про транспортні засоби від МВС, що показує реальний стан авторинку України.
Лише за перші три місяці опубліковану інформацію завантажили 20 тис. разів.
На базі відомостей про транспортні засоби «Texty.org.ua» та «Opendatabot» створили сервіси,
якими громадяни скористалися понад 1 млн разів.
Окремо слід відзначити відкриття у 2018 році даних щодо ліцензій транспортних засобів
на перевезення пасажирів і вантажів. Тепер за кілька секунд кожен охочий за номером
транспортного засобу може перевірити, чи він користується послугами одного з 42 794 легальних перевізників.
З ініціативи Міністерства Фінансів України на порталі почали публікуватися дані 9 603 місцевих бюджетів.
Тепер кожен громадянин може контролювати використання бюджетних коштів на рівні області та села.
Також контрольні державні органи відкрили дані про понад 143 400 перевірок бізнесу,
запланованих на 2019 рік. Тепер будь-який власник бізнесу може дізнатися,
коли перевірка постукає у двері його компанії.
У сфері екології найбільший вплив мало відкриття даних державного моніторингу поверхневих вод
Державним агентством водних ресурсів.
Дані оприлюднені за 16 основними показниками із 445 пунктів збору води на річках за останні п'ять років.
Хоча дані про якість води вважаються одним з найважливіших наборів для суспільства,
за кордоном вони залишаються доволі закритими.
Згідно з останнім випуском «Global Open Data Index», ці дані публікують лише 15 країн світу.
Завдяки відкритим даним Міністерства Екології та природних ресурсів щодо дозвільних документів
та процедур промислових та інших забруднювачів довкілля активісти з Дніпра створили перший екологічний бот «SaveEcoBot».

\textbf{Національний конкурс.}
Відкриті дані мають найбільший вплив, коли їх використовують для створення продуктів та сервісів для економічного розвитку та інновацій.

У 2018 році в Україні вдруге відбувся національний конкурс інноваційних IT-проектів на основі відкритих даних «Open Data Challenge».
Заявки на участь у заході подали 190 команд, 15 з них пройшли інкубацію, а шість команд-переможців розділили фінансову допомогу на загальну суму 2,5 млн грн.
Переможці конкурсу роблять суттєвий внесок у розвиток економіки країни.
Так, проект «Monitor Estate» дозволяє перевіряти ризики купівлі новобудови у Києві або Львові.
«NORA» аналізує дані з відкритих джерел і виявляє неочевидні зв'язки між учасниками будівельного ринку, 
а «LvivCityHelper» інформує львів'ян про ремонти та обслуговування будинків.
Як показала практика, найбільший попит на відкриті дані — у сферах інфраструктури, будівництва та екології.

Третій цикл «Open Data Challenge»  почнеться у лютому 2019 року,
тож кожен може подати заявку та виграти фінансову підтримку проекту на основі відкритих даних.
У 2017 році уряд схвалив оновлену постанову №835, що регулює відкриття понад 600 наборів даних.
Попит на відкриті дані росте, тому очікується ухвалення третьої редакції постанови та розширення кількості наборів.
Публікація даних про транспорт та охорону здоров'я може стати найбільшим проривом,
у тому числі завдяки великому антикорупційному потенціалу.
Ключовим пріоритетом залишається стимулювання використання даних для розвитку економіки.
Одну з провідних ролей відіграватиме третій цикл «Open Data Challenge».

\section{Публічна інформація у формі відкритих даних}

Закон України «Про доступ до публічної інформації» надав
публічній інформації статус доступності, але передбачав спочатку
лише доступ до публічної інформації за запитами. Зміни до цього
закону, пов'язані з виокремленням нового типу публічної інформації,
а саме, публічної інформації у формі відкритих даних.
Тепер документи, в яких знаходяться відкриті дані, обов'язково
повинні бути електронними, оприлюдненими та
придатними для повторного використання з можливістю наступного
використання з метою проведення наукових досліджень, забезпечення
інновації, запровадження бізнес-проектів, забезпечення підзвітності і
суспільного контролю за органами публічної влади.

Можна виділити такі основні мотиви відкриття державних
даних: запобігання корупції, надання послуг та забезпечення інновацій,
посилення громадської участі та контролю,
укріплення правопорядку та законності.
Концепція відкритих даних активно підтримується та розвивається міжнародними
ініціативами та організаціями, зокрема, в рамках ініціативи
партнерство «Відкритий Уряд», до якої Україна приєдналась у 2011
році.

Відкриті дані повинні бути придатні до повторного використання без обмежень авторського
права, патентів та інших механізмів контролю \cite{OpenDataLovers}.
Більшість країн світу використовує вільну ліцензію некомерційної
організації «Creative Commons», яка розробила вільні та відкриті
публічні ліцензії, за допомогою яких автори та правовласники можуть
поширювати свою інформаційну продукцію, а користувачі контенту
легально їх використовувати.

Відкриті ліцензії «Creative Commons» дозволяють повторне використання
інформації державного сектора без необхідності розробляти
та оновлювати на замовлення ліцензії на національному рівні.
Таку ліцензію використовують Австралія, Бразилія, Великобританія, Франція, Нідерланди.

Хартія відкритих даних формулює шість
основних принципів роботи з відкритими даними та визначає пріоритетні
сфери, в яких відкриті дані матимуть найбільший ефект.

Відкриті дані виступають як фундамент відкритого публічного управління,
сприяють прозорості роботи органів публічної влади, формується
база для громадського контролю та створюються нові послуги для
громадян та бізнесу.

Одним з ключових елементів концепції відкритих даних є консолідація державних даних,
тому єдиний державний портал відкритих даних виступає ключовим елементом
та є засобом консолідації цих даних \cite{GovPublicDataLaw}.
Завданням порталу є систематизація та зручне представлення
даних відповідно до інтересів та запитів користувачів.