\chapter{Відкриті дані}

\section{Поняття відкритих даних}


Переважна більшість організацій користується визначенням, якe було дано всесвітньою організацією «Open Knowledge Foundation».
Це визначення відкритості подає точне значення терміну «відкрите» стосовно знань,
сприяючи стійким громадам, в яких кожен може взяти участь
і де можливість взаємодії досягає максимуму.



Повне значення відкритості є надто чітким для цієї роботи,
виділимо з визначення основні риси відкритості:


\textbf{Доступність.}
Дані повинні бути доступні повністю, їх доступність повинна бути
у межах розумних витрат, переважно шляхом з авантаження через Інтернет.
Дані також повинні бути доступними у зручній і змінюваній формі.

\textbf{Повторне використання та розповсюдження.}
Дані повинні надаватися на умовах, що дозволяють повторне
використання та розповсюдження, включаючи
перемішування з іншими наборами даних.

\textbf{Загальна участь.}
Кожен повинен мати змогу використовувати та розповсюджувати.
Будь-які обмеження щодо сфер діяльності або проти осіб або груп повинні бути відсутніми.
Наприклад, не комерційні обмеження, які запобігають комерційному використанню,
або обмежується використання для певних цілей (наприклад, тільки в освіті), не допускаються.

Якщо вам цікаво, чому так важливо розуміти,
які відкриті засоби та чому використовується це визначення, є проста відповідь: сумісність.

Сумісність означає здатність різних систем та організацій взаємодіяти разом.
У даному випадку це можливість взаємодіяти або змішувати різні набори даних.

Сумісність є дуже важливою, оскільки дозволяє різним компонентам працювати разом.
Здатність до складання компонентів і «підключення» компонентів має важливе
значення для побудови великих, складних систем.
Без сумісності це стає неможливим, про що свідчить найвідоміший міф про
Вавилонську вежу, де проблема взаємодії призвела до повного краху.

\section{Історія відкритих даних}

\section{Відкриті дані в Україні}
